\documentclass[xcolor=dvipsnames]{beamer}

% Standard packages
\usepackage{amsmath,amsthm,amsfonts,mathabx}
\usepackage{babel}
\usepackage[utf8]{inputenc}
\usepackage[T1]{fontenc}
\usepackage{graphicx}
\usepackage{listings}

% Basic Julia definition for lstlisting
\input juliadef.tex

\lstdefinelanguage{LDRjl}{
  language=Julia,
  morekeywords=[2]{LinearDecisionRules,LDRModel,Uncertainty,SolvePrimal,SolveDual,get_decision},
  keywordstyle=[2]\bfseries\color{red},
}

\definecolor{mygreen}{RGB}{34,139,34}
\lstset{%
    language         = LDRjl,
    basicstyle       = \ttfamily,
    keywordstyle     = \bfseries\color{blue},
    stringstyle      = \color{magenta},
    commentstyle     = \color{mygreen},
    showstringspaces = false,
    breaklines       = true,
    extendedchars    = true,
    literate={≈}{{$\approx$}}1%
}

% Setup appearance and colors
\usetheme{Boadilla}
\setbeamertemplate{page number in head/foot}[appendixframenumber]

\setbeamercovered{invisible}
\usefonttheme[onlymath]{serif}
\setbeamertemplate{enumerate items}[default]
\definecolor{mpl-c2}{HTML}{2ca02c}

\setbeamercolor{emph}{fg=blue}
\renewcommand<>{\emph}[1]{%
  {\usebeamercolor[fg]{emph}\only#2{\itshape}#1}%
}

\newcommand{\blue}[1]{\textcolor{blue}{#1}}
\newcommand{\alertt}[1]{\alert{\texttt{#1}}}

\DeclareMathOperator{\Tr}{Tr}

% Commands


\begin{document}

\title[LinearDecisionRules.jl]
      {\texttt{LinearDecisionRules.jl}}
\author[Bernardo Costa]
       {Bernardo Freitas Paulo da Costa (UFRJ \& FGV)\\[2ex]
       with Joaquim Garcia (PSR)}
\date[JuMP-dev 24]{JuMP-dev 2024, Montréal}

% Multiple logos
\titlegraphic{
  % \includegraphics[width=2cm]{/media/sf_Share_VMs/Logos/UFRJ/ufrj-horizontal-cor-rgb-telas.png}
  % \hspace{2cm}
  % \includegraphics[width=2cm]{/media/sf_Share_VMs/Logos/Faperj-Logo.png}
  % \hspace{2cm}
  % \includegraphics[width=2cm]{/media/sf_Share_VMs/Logos/FGV_EMAp_logo.png}
  % EDPonts / CERMICS
}

\begin{frame}
\titlepage
\end{frame}

\section{Introduction}

\subsection{Setting}

\begin{frame}{Stochastic Programming}
  A simple model for stochastic programming:
  \[ \begin{array}{rl}
    \min        & \mathbb{E}\left[ c^\top x \right] \\[0.5ex]
    \text{s.t.} & A x = b, \\
    & x \geq 0.
  \end{array} \]
  where
  \begin{itemize}
    \item $x$ is the \alert{decision}, subject to (random) constraints;
    \item $c$ are the (possibly random) \alert{costs};
  \end{itemize}
\end{frame}

\begin{frame}{Linear Decision Rules}
  We write the uncertain parameters as \only<3->{\alert{linear} }functions of an underlying random vector $\xi$, and allow for the decision to be taken \emph{after observing the realization of $\xi$}:
  \[ \begin{array}{rll}
    \min        & \mathbb{E}\left[
      \alt<1-2>{c(\xi)^\top}{\xi^\top C^\top}
      \alt<1-4>{x(\xi)}{X \xi} \right] \\[0.5ex]
    \text{s.t.} & A \alt<1-4>{x(\xi)}{X \xi} = \alt<1-2>{b(\xi)}{B \xi} & \forall \xi \in \Xi, \\
    & \alt<1-4>{x(\xi)}{X \xi} \geq 0 & \forall \xi \in \Xi.
  \end{array} \]
  \pause
  We assume that the constraint matrix is \emph{deterministic}.
  \pause[4]

  We then posit a \alert{linear decision rule} for $x$:
  \[
    x(\xi) = X \xi.
  \]
  \pause[6]

  This reduces the flexibility of the ``wait-and-see'' decision, but allows for a \emph{more tractable} optimization problem.
\end{frame}

\begin{frame}{Linear Decision Rules --- Reformulation}
  If the uncertainty set $\Xi$ is given as the polytope $\{\, \xi : W \xi \geq h \,\}$, we can rewrite the optimization problem as a linear program over the decision rule matrix $X$ and auxiliary variables $\Lambda$ (for the positivity constraints):
  \[ \begin{array}{rl}
    \min\limits_{X, \Lambda} & \Tr\left(\mathbb{E}\left[\xi \xi^\top \right] C^\top X \right) \\[0.5ex]
    \text{s.t.} & A X = B, \\
    & X = \Lambda W, \ \Lambda h \geq 0, \ \Lambda \geq 0.
  \end{array} \]
\end{frame}

\section{Usage}

\begin{frame}{The \texttt{LinearDecisionRules.jl} Package}
  The package \texttt{LinearDecisionRules.jl} provides a JuMP extension for modeling Stochastic Programming problems with linear decision rules.
  \pause

  \begin{block}{Usage}
    \begin{enumerate}[<+->]
      \item We introduce \alertt{LDRModel} as an extension of \texttt{JuMP.Model};
      \item The \texttt{@variable} macro is \alert{extended} to allow for the declaration of \emph{uncertainties} as variables in the model;
      \item Attributes \alertt{SolvePrimal()} and \alertt{SolveDual()} enable and disable the optimization of primal and dual LDR reformulations.
      \item We provide \alertt{get\_decision()} to extract the coefficients of the decision rule matrix $X$ in the original variables and uncertainties. A keyword argument \alertt{dual} is used for querying dual decision rule.
    \end{enumerate}
  \end{block}
\end{frame}

\begin{frame}[fragile]{A toy (energy!) example}
  \small
  \begin{lstlisting}{language=LDRjl}
    using JuMP, LinearDecisionRules
    using Ipopt, Distributions

    demand = 0.3
    initial_volume = 0.5

    m = LDRModel()
    @variable(m, vi == initial_volume)
    @variable(m, 0 <= vf <= 1)
    @variable(m, gh >= 0.0)
    @variable(m, gt >= 0.0)
    @variable(m, 0 <= inflow <= 0.2, Uncertainty, distribution=Uniform(0, 0.2))

    @constraint(m, balance, vf == vi - gh + inflow)
    @constraint(m, gt + gh == demand)

    @objective(m, Min, gt^2 + vf^2/2 - vf)
  \end{lstlisting}
\end{frame}

\begin{frame}[fragile]{A toy example (cont.)}
  \small
  \begin{lstlisting}{language=LDRjl}
    # Solve the primal LDR
    set_attribute(m, SolvePrimal(), true)
    set_attribute(m, SolveDual(), false)
    set_optimizer(m, Ipopt.Optimizer)
    optimize!(m)

    # Get the decision rule
    get_decision(m, vf)         # Constant term
    get_decision(m, vf, inflow) # Linear coefficient
    
    # Some checks
    @test get_decision(m, gh) + get_decision(m, gt) ≈ demand atol=1e-6
    @test get_decision(m, gh, inflow) + get_decision(m, gt, inflow) ≈ 0 atol=1e-6

    @test get_decision(m, vi) ≈ initial_volume atol=1e-6
    @test get_decision(m, vi, inflow) ≈ 0 atol=1e-6
  \end{lstlisting}
\end{frame}

\begin{frame}{Package structure}
  \centering \includegraphics[width=0.9\textwidth]{figures/diagram.pdf}
\end{frame}

\begin{frame}{Next steps}
  Handle \emph{correlated uncertainties}:
  \begin{itemize}
    \item The current model allows for independent uncertainties, and $\Xi$ is the product of their support;
    \item We could allow for a general form as the product of independent \emph{vector uncertainties}.
  \end{itemize}
  \pause

  \emph{Multistage} decision rules:
  \begin{itemize}
    \item 2-stage optimization: a \emph{here-and-now} decision $x_0$ which does not depend on uncertainty;
      \pause
    \item In general, decisions $x_t$ can only depend on \emph{observed} uncertainties $\xi_1, \ldots, \xi_t$;
    \item Will benefit from correlated uncertainties to model more complex processes.
  \end{itemize}
\end{frame}

\begin{frame}
  \centering \includegraphics[width=0.5\textwidth]{figures/ldrgit.pdf}

  \Huge Questions?
\end{frame}


\end{document}


% vim:set spelllang=en:

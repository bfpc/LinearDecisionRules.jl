%%%%%%%%%%%%%%%%%%%%%%%%%%%%%%%%%%%%%%%%%
% baposter Landscape Poster
% LaTeX Template
% Version 1.0 (11/06/13)
%
% baposter Class Created by:
% Brian Amberg (baposter@brian-amberg.de)
%
% This template has been downloaded from:
% http://www.LaTeXTemplates.com
%
% License:
% CC BY-NC-SA 3.0 (http://creativecommons.org/licenses/by-nc-sa/3.0/)
%
%%%%%%%%%%%%%%%%%%%%%%%%%%%%%%%%%%%%%%%%%

%----------------------------------------------------------------------------------------
%	PACKAGES AND OTHER DOCUMENT CONFIGURATIONS
%----------------------------------------------------------------------------------------

\documentclass[a0paper,fontscale=0.285]{baposter} % Adjust the font scale/size here

\usepackage{graphicx} % Required for including images
\graphicspath{{figures/}} % Directory in which figures are stored

\usepackage{amsmath,amssymb}

\usepackage{booktabs} % Top and bottom rules for tables
\usepackage{enumitem} % Used to reduce itemize/enumerate spacing
\usepackage{palatino} % Use the Palatino font
\usepackage[font=small,labelfont=bf]{caption} % Required for specifying captions to tables and figures

\usepackage{multicol} % Required for multiple columns
\setlength{\columnsep}{1.5em} % Slightly increase the space between columns
\setlength{\columnseprule}{0mm} % No horizontal rule between columns

\usepackage{tikz} % Required for flow chart
\usetikzlibrary{shapes,arrows} % Tikz libraries required for the flow chart in the template

\newcommand{\compresslist}{ % Define a command to reduce spacing within itemize/enumerate environments, this is used right after \begin{itemize} or \begin{enumerate}
\setlength{\itemsep}{1pt}
\setlength{\parskip}{0pt}
\setlength{\parsep}{0pt}
}

\definecolor{lightblue}{rgb}{0.145,0.6666,1} % Defines the color used for content box headers

\usepackage{listings}

% Basic Julia definition for lstlisting
\input juliadef.tex

\lstdefinelanguage{LDRjl}{
  language=Julia,
  morekeywords=[2]{LinearDecisionRules,LDRModel,Uncertainty,SolvePrimal,SolveDual,get_decision,FirstStage,BreakPoints},
  keywordstyle=[2]\bfseries\color{red},
}

\definecolor{mygreen}{RGB}{34,139,34}
\lstset{%
    language         = LDRjl,
    basicstyle       = \ttfamily,
    keywordstyle     = \bfseries\color{blue},
    stringstyle      = \color{magenta},
    commentstyle     = \color{mygreen},
    showstringspaces = false,
    breaklines       = true,
    extendedchars    = true,
    literate={≈}{{$\approx$}}1%
}

\DeclareMathOperator{\Tr}{Tr}
\newcommand{\alert}[1]{\textcolor{blue}{#1}}
\newcommand{\alertt}[1]{\alert{\texttt{#1}}}


\begin{document}

\begin{poster}
{
headerborder=closed, % Adds a border around the header of content boxes
colspacing=1em, % Column spacing
bgColorOne=white, % Background color for the gradient on the left side of the poster
bgColorTwo=white, % Background color for the gradient on the right side of the poster
borderColor=lightblue, % Border color
headerColorOne=black, % Background color for the header in the content boxes (left side)
headerColorTwo=lightblue, % Background color for the header in the content boxes (right side)
headerFontColor=white, % Text color for the header text in the content boxes
boxColorOne=white, % Background color of the content boxes
textborder=roundedleft, % Format of the border around content boxes, can be: none, bars, coils, triangles, rectangle, rounded, roundedsmall, roundedright or faded
eyecatcher=true, % Set to false for ignoring the left logo in the title and move the title left
headerheight=0.1\textheight, % Height of the header
headershape=roundedright, % Specify the rounded corner in the content box headers, can be: rectangle, small-rounded, roundedright, roundedleft or rounded
headerfont=\Large\bf\textsc, % Large, bold and sans serif font in the headers of content boxes
%textfont={\setlength{\parindent}{1.5em}}, % Uncomment for paragraph indentation
linewidth=2pt % Width of the border lines around content boxes
}
%----------------------------------------------------------------------------------------
%	TITLE SECTION 
%----------------------------------------------------------------------------------------
%
{\includegraphics[height=2em]{emap_logo.pdf}} % First university/lab logo on the left
{\bf\textsc{LinearDecisionRules.jl}\vspace{0.5em}} % Poster title
{\textsc{Bernardo Costa \& Joaquim Garcia}} % Author names and institution
{\includegraphics[height=4em]{logo_hpsc.png}} % Second university/lab logo on the right

%----------------------------------------------------------------------------------------
%	OBJECTIVES
%----------------------------------------------------------------------------------------

\headerbox{Setting}{name=setting,column=0,row=0}{

Several optimization problems need to take \emph{uncertainty} into account.
A simple model is stochastic \emph{linear} programming, where uncertainty
is modeled as a random variable $\xi$:
\[ \begin{array}{rl}
  \min        & \mathbb{E}\left[ c(\xi)^\top x(\xi) \right] \\[0.5ex]
  \text{s.t.} & A x(\xi) = b(\xi), \\
  & x(\xi) \geq 0.
\end{array} \]

There
\begin{itemize}
  \item $x$ is the \textbf{decision}, depending on the realization $\xi \in \Xi$,
  \item $c$ are the (possibly random) \textbf{costs};
\end{itemize}

}

%----------------------------------------------------------------------------------------
%	INTRODUCTION
%----------------------------------------------------------------------------------------
\headerbox{LDR Reformulation}{name=reformulation,column=1,row=0}{
  We posit $x(\xi) = X \cdot \xi$ as a \textbf{Linear Decision Rule} for the decision
  $x$.

  If the uncertainty set $\Xi$ is given as the polytope $\{\, \xi : W \xi \geq h \,\}$, we can rewrite the optimization problem as a linear program over the decision rule matrix $X$ and auxiliary variables $\Lambda$ (for the positivity constraints):
  \[ \begin{array}{rl}
    \min\limits_{X, \Lambda} & \Tr\left(\mathbb{E}\left[\xi \xi^\top \right] C^\top X \right) \\[0.5ex]
    \text{s.t.} & A X = B, \\
    & X = \Lambda W, \ \Lambda h \geq 0, \ \Lambda \geq 0.
  \end{array} \]

  This involves manipulating constraint / cost data to build the matrices $A$, $B$, $C$, $W$ and~$h$.

  However, constructing the expectation $\mathbb{E}\left[\xi \xi^\top \right]$ depends on the \emph{probability distribution of $\xi$}.
}

\headerbox{Objectives}{name=objectives,column=2,row=0}{

\begin{enumerate}\compresslist
  \item Provide a \texttt{JuMP} interface for modeling uncertain problems;
  \item Simplify the construction of Linear Decision Rules for stochastic optimization;
  \item Allow for higher-complexity models such as piecewise-linear DR's.
\end{enumerate}

% \vspace{0.3em} % When there are two boxes, some whitespace may need to be added if the one on the right has more content
%
}

%----------------------------------------------------------------------------------------
%	Package Features
%----------------------------------------------------------------------------------------
\headerbox{Features}{name=features,column=2,below=objectives}{

  \begin{enumerate}\compresslist
    \item Some decision variables can be labeled as \alertt{FirstStage} to accomodate ``here-and-now'' decisions;
    \item First-stage decisions can even be \emph{integer} or \emph{binary};
      \medskip

    \item Uncertainties have bounded support, and can use \alertt{Distributions.} \alertt{truncated};
    \item Univariate distributions can be \emph{lifted} to piecewise-LDRs: more flexibility via \alertt{BreakPoints()}.
    \item \emph{Efficient} formulas for the second-moment matrix $\mathbb{E}\left[\xi \xi^\top \right]$.
      \medskip

    \item Multivariate distributions (such as \alertt{MvNormal}) can be truncated explicitly via \texttt{@constraint} statements and lead to \emph{rejection sampling} for estimating the second-moment matrix.
    \item This mechanism works for imposing arbitrary inequality constraints on the uncertainties.
  \end{enumerate}

}

%----------------------------------------------------------------------------------------
%	First example: Hydro dispatch
%----------------------------------------------------------------------------------------

\begin{posterbox}[name=example1,column=0,span=2,below=reformulation]
{Simple hydrothermal dispatch example}

  \begin{multicols}{2}
  \small
  \begin{lstlisting}{language=LDRjl}
using JuMP, LinearDecisionRules
using Ipopt, Distributions

demand = 0.3
initial_volume = 0.5

m = LDRModel(Ipopt.Optimizer)
@variable(m, vi == initial_volume)
@variable(m, 0 <= vf <= 1)
@variable(m, gh >= 0.0)
@variable(m, gt >= 0.0)
@variable(m, inflow, Uncertainty, distribution=Uniform(0, 0.2))

@constraint(m, balance, vf == vi - gh + inflow)
@constraint(m, gt + gh == demand)
@objective(m, Min, gt^2 + vf^2 - vf)

# Solve the primal LDR
set_attribute(m, SolvePrimal(), true)
set_attribute(m, SolveDual(), false)
optimize!(m)

# Get constant term and linear coefficient in decision rule
get_decision(m, vf)
get_decision(m, vf, inflow)
  \end{lstlisting}
  \end{multicols}
\end{posterbox}

%----------------------------------------------------------------------------------------
%	Second example: Stoch UC
%----------------------------------------------------------------------------------------

\begin{posterbox}[name=example2,column=0,span=2,below=example1]
{Stochastic Unit Commitment}
  % This block's bottom aligns with the bottom of the conclusion block
  { \footnotesize
  \begin{lstlisting}{language=LDRjl}
# Uncertainty in the demand (wrt. baseline)
dist_d = Normal.(0.0, sigma_d)
dist_d = truncated.(dist_d, -3 * sigma_d, 3 * sigma_d)

# Model modifications
# plant on/off & startup indicator
@variable(uc_ldr, 0 <= x[i in Gens, t in 0:T] <= 1, LDR.FirstStage, integer=true)
@variable(uc_ldr, 0 <= y[i in Gens, t in 1:T] <= 1, LDR.FirstStage, integer=true)
# Variable for uncertainty in demand
@variable(uc_ldr, extra_demand[t in 1:T] in LDR.Uncertainty(distribution=dist_d[t]))
# generation / flow / ... decision variables will be LDRs of extra_demand
# load balance constraint
@constraint(uc_ldr, LoadBalance[t in 1:T], sum(g[i, t] for i in Gens) + deficit[t] == demand[t] + extra_demand[t])
  \end{lstlisting} }

  \begin{center}
    \begin{tabular}{lrr}
      \toprule
      Model & In-sample & Out-of-sample \\ \midrule
      Deterministic  & 542507 & 550183 \\
      SAA            & 542425 & 542702 \\
      LDR Scenarios  & 542654 & 542702 \\
      LDR Polyhedral & 543359 & 542698 \\ \bottomrule
    \end{tabular}
    \captionof{table}{Several models for StochUC}
  \end{center}

  The SAA model is the most flexible one, but all stochastic models perform well
  out-of-sample.

  The LDR model with scenarios can lead to \emph{negative} dispatch during out-of-sample simulation.

\end{posterbox}

%----------------------------------------------------------------------------------------
%	Next steps
%----------------------------------------------------------------------------------------

\headerbox{Next steps}{name=nextsteps,column=0,span=2,above=bottom,below=example2}{

\begin{multicols}{2}
  Improve \emph{correlated uncertainties}, e.g. confidence ellipsoids for
  \texttt{MvNormal} distributions.

  \emph{Multistage} decision rules, generalizing \alertt{FirstStage} decisions.

  \emph{Performance} for building large-scale problems

\end{multicols}
}

%----------------------------------------------------------------------------------------
%	CONTACT INFORMATION
%----------------------------------------------------------------------------------------

\headerbox{More Information}{name=more,column=2,above=bottom}{

  \centering \includegraphics[width=0.5\textwidth]{../ICS_2025/figures/ldrgit.pdf}
}


%----------------------------------------------------------------------------------------

\end{poster}

\end{document}

A typical 2-stage formulation is
\[ \begin{array}{rl}
    \min        & c^\top x + \mathbb{E}\left[ d^\top y\right] \\[0.5ex]
    \text{s.t.} & A x = b, \ \  x \geq 0 \\
    & T x + W y = h \ \ y \geq 0.
\end{array} \]
where
\begin{itemize}
  \item $x$ is the \textbf{strategic decision}
  \item $y$ is the \textbf{recourse decision}, subject to (random) constraints;
  \item $c$ and $d$ are \textbf{costs} ($d$ can be random);
\end{itemize}


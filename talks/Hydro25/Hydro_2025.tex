\documentclass[xcolor=dvipsnames,aspectratio=169]{beamer}

% Standard packages
\usepackage{amsmath,amsthm,amsfonts,mathabx}
\usepackage{babel}
\usepackage[utf8]{inputenc}
\usepackage[T1]{fontenc}
\usepackage{graphicx}
\graphicspath{{figures/}{../ICS_2025/figures}} % Directory in which figures are stored

\usepackage{listings}

% Basic Julia definition for lstlisting
\input juliadef.tex

\lstdefinelanguage{LDRjl}{
  language=Julia,
  morekeywords=[2]{LinearDecisionRules,LDRModel,Uncertainty,SolvePrimal,SolveDual,get_decision,FirstStage,BreakPoints},
  keywordstyle=[2]\bfseries\color{red},
}

\definecolor{mygreen}{RGB}{34,139,34}
\lstset{%
    language         = LDRjl,
    basicstyle       = \ttfamily,
    keywordstyle     = \bfseries\color{blue},
    stringstyle      = \color{magenta},
    commentstyle     = \color{mygreen},
    showstringspaces = false,
    breaklines       = true,
    extendedchars    = true,
    literate={≈}{{$\approx$}}1%
}

% Setup appearance and colors
\usetheme{Boadilla}
\setbeamertemplate{page number in head/foot}[appendixframenumber]

\setbeamercovered{invisible}
\usefonttheme[onlymath]{serif}
\setbeamertemplate{enumerate items}[default]
\definecolor{mpl-c2}{HTML}{2ca02c}
\definecolor{darkolivegreen}{rgb}{0.33, 0.42, 0.18}
\definecolor{darkorchid}{rgb}{0.6, 0.2, 0.8}

\setbeamercolor{emph}{fg=blue}
\renewcommand<>{\emph}[1]{%
  {\usebeamercolor[fg]{emph}\only#2{\itshape}#1}%
}

\newcommand{\blue}[1]{\textcolor{blue}{#1}}
\newcommand{\alertt}[1]{\alert{\texttt{#1}}}

\DeclareMathOperator{\Tr}{Tr}

% Commands


\begin{document}

\title[LinearDecisionRules.jl]
      {\texttt{LinearDecisionRules.jl}}
\author[Bernardo Costa]
       {Bernardo Freitas Paulo da Costa (FGV EMAp)\\[2ex]
       with Joaquim Garcia (PSR)}
\date[HPSC 2025]{Hydroscheduling 2025, Rio de Janeiro}

% Multiple logos
\titlegraphic{
  \includegraphics[height=2ex]{emap_logo.pdf}
  \hspace{2em}
  \includegraphics[height=4ex]{logo_hpsc.png}
}

\begin{frame}
\titlepage
\end{frame}

\section{Introduction}

\subsection{Setting}

\begin{frame}{What is a Linear Decision Rule?}
  \def\xiC{\textcolor{blue}{\xi^\top C^\top}}
  \def\Bxi{\textcolor{blue}{B \xi}}
  \def\Cxi{\textcolor{blue}{C \xi}}
  \def\Xxi{\textcolor{darkorchid}{X \xi}}
  \begin{columns}[t]
    \begin{column}{0.5\textwidth}
      Fully-flexible Stochastic optimization:
      \begin{itemize}\itemsep0pt
        \item The ``recourse'' decision $x(\xi)$ is independently chosen for each scenario.
          \uncover<2->{%
          \begin{itemize}
            \item The SAA problem may be large if we need several scenarios.
          \end{itemize}}
      \end{itemize}
      \bigskip
      \bigskip

      \[ \begin{array}{rll}
        \min        & \mathbb{E}\left[ c(\xi)^\top x(\xi) \right] \\[0.5ex]
        \text{s.t.} & A x(\xi) = b(\xi) & \forall \xi \in \Xi, \\
                    &   x(\xi) \geq 0   & \forall \xi \in \Xi.
      \end{array} \]
    \end{column}
    \pause[3]
    \begin{column}{0.5\textwidth}
      % Motto: write everything as \emph{linear} functions of the uncertainty $\xi \in \Xi$.
      Linear Decision Rule:
      \begin{itemize}\itemsep0pt
        \item Write $x(\xi) = X \cdot \xi$.

          \uncover<4>{
          \begin{itemize}
            \item Choose a parametrization where $c(\xi) = \Cxi$ and $b(\xi) = \Bxi$ are \emph{linear} in $\xi$.
            \item Simpler dependency, suboptimal, but easier to optimize (and interpret).
          \end{itemize}}
      \end{itemize}
      \medskip

      \[ \begin{array}{rll}
        \min        & \mathbb{E}\left[ \alt<3>{c(\xi)}{\xiC} \Xxi \right] \\[0.5ex]
        \text{s.t.} & A \Xxi = \alt<3>{b(\xi)}{\Bxi} & \forall \xi \in \Xi, \\
                    &   \Xxi \geq 0                  & \forall \xi \in \Xi.
      \end{array} \]
    \end{column}
  \end{columns}
\end{frame}

\begin{frame}{Linear Decision Rules --- LP Reformulation}
  \begin{itemize}
    \item If $\Xi = \{\, \xi : W \xi \geq h \,\}$, rewrite as a \emph{deterministic} LP over the decision rule matrix $X$ and auxiliary variables $\Lambda$ (for the positivity constraints):
      \[ \begin{array}{rll}
        \min        & \mathbb{E}\left[ \xi^\top C^\top X \xi \right] \\[0.5ex]
        \text{s.t.} & A X \xi = B \xi & \forall \xi \in \Xi, \\
                    &   X \xi \geq 0  & \forall \xi \in \Xi.
      \end{array}
      \quad \to \quad
      \begin{array}{rl}
        \min\limits_{X, \Lambda} & \Tr\left(\mathbb{E}\left[\xi \xi^\top \right] C^\top X \right) \\[0.5ex]
        \text{s.t.} & A X = B, \\
        & X = \Lambda W, \ \Lambda h \geq 0, \ \Lambda \geq 0.
      \end{array} \]
      \pause\medskip

    \item Involves manipulating the constraint / cost data to build matrices $A$, $B$, $C$, $W$ and~$h$.
      \pause\medskip

    \item The expectation $\mathbb{E}\left[\xi \xi^\top \right]$ depends on the \emph{probability distribution of $\xi$}.
  \end{itemize}
\end{frame}

\section{Usage}

\begin{frame}{The \texttt{LinearDecisionRules.jl} Package}
  \texttt{JuMP} extension for modeling problems with LDR's.
  \pause

  \begin{block}{Basics}
    \begin{enumerate}
      \item \alertt{LDRModel} is an extension of \texttt{JuMP.Model};
      \item The \alertt{@variable} macro is \alert{extended} to declare \emph{uncertainties} in the model;
      \begin{itemize}
        \item Probabilities can come from \texttt{Distributions.jl};
        \item \texttt{DiscreteMvNonParametric} can be used for explicit scenarios.
      \end{itemize}
      \pause\medskip

      \item \alertt{SolvePrimal()} and \alertt{SolveDual()} enable and disable the optimization of primal and dual LDR reformulations.
      \item \alertt{get\_decision()} extracts the coefficients from the decision rule matrix $X$ in the original variables and uncertainties. A keyword argument \alertt{dual} is used for dual decision rules.
    \end{enumerate}
  \end{block}
\end{frame}

\begin{frame}[fragile]{A toy example}
  \small
  \begin{lstlisting}{language=LDRjl}
    using JuMP, LinearDecisionRules
    using Ipopt, Distributions

    demand = 0.3
    initial_volume = 0.5

    m = LDRModel()
    @variable(m, vi == initial_volume)
    @variable(m, 0 <= vf <= 1)
    @variable(m, gh >= 0.0)
    @variable(m, gt >= 0.0)
    @variable(m, inflow, Uncertainty, distribution=Uniform(0, 0.2))

    @constraint(m, balance, vf == vi - gh + inflow)
    @constraint(m, gt + gh == demand)

    @objective(m, Min, gt^2 + vf^2/2 - vf)
  \end{lstlisting}
\end{frame}

\begin{frame}[fragile]{A toy example (cont.)}
  \small
  \begin{lstlisting}{language=LDRjl}
    # Solve the primal LDR
    set_attribute(m, SolvePrimal(), true)
    set_attribute(m, SolveDual(), false)
    set_optimizer(m, Ipopt.Optimizer)
    optimize!(m)

    # Get the decision rule
    get_decision(m, vf)         # Constant term
    get_decision(m, vf, inflow) # Linear coefficient
  \end{lstlisting}
\end{frame}

\begin{frame}{Features}
  \begin{enumerate}
    \item Some decision variables can be labeled as \alertt{FirstStage} to accomodate ``here-and-now'' decisions;
    \item First-stage decisions can even be \emph{integer} or \emph{binary};
      \pause \medskip

    \item Uncertainties have bounded support, and \alertt{Distributions.truncated} works well with the package;
    \item Univariate distributions can be \emph{lifted} to piecewise-LDRs: more flexibility via the \alertt{BreakPoints()} attribute.
    \item \emph{Efficient} formulas whenever possible for the second-moment matrix $\mathbb{E}\left[\xi \xi^\top \right]$.
      \pause \medskip

    \item Multivariate distributions (\alertt{MvNormal}, \ldots) can be truncated explicitly via \texttt{@constraint} statements and lead to \emph{rejection sampling} for estimating the second-moment matrix.
    \item Works for imposing arbitrary inequality constraints on the uncertainties.
  \end{enumerate}
\end{frame}

\begin{frame}[fragile]{Stochastic Unit Commitment}
  \small
  \begin{lstlisting}{language=LDRjl}
    # Uncertainty in the demand (wrt. baseline)
    dist_d = Normal.(0.0, sigma_d)
    dist_d = truncated.(dist_d, -3 * sigma_d, 3 * sigma_d)

    # Model modifications
    # plant on/off & startup indicator
    @variable(uc_ldr, 0 <= x[i in Gens, t in 0:T] <= 1, LDR.FirstStage, integer=true)
    @variable(uc_ldr, 0 <= y[i in Gens, t in 1:T] <= 1, LDR.FirstStage, integer=true)
    # variable Demand
    @variable(uc_ldr, extra_demand[t in 1:T] in LDR.Uncertainty(distribution=dist_d[t]))
    # load balance
    @constraint(uc_ldr, LoadBalance[t in 1:T], sum(g[i, t] for i in Gens) + deficit[t] == demand[t] + extra_demand[t])
  \end{lstlisting}
\end{frame}

\begin{frame}{Stochastic Unit Commitment}
  \begin{table}
    \caption{Several models for StochUC}
    \begin{tabular}{lrr}
      Model & In-sample & Out-of-sample \\ \hline
      Deterministic  & 542507 & 550183 \\
      SAA            & 542425 & 542702 \\
      LDR Scenarios  & 542654 & 542702 \\
      LDR Polyhedral & 543359 & 542698 \\ \hline
    \end{tabular}
  \end{table}
  \pause

  \begin{itemize}
    \item All Stochastic models have similar out-of-sample performance
    \item The LDR model with scenarios can lead to \emph{negative} out-of-sample dispatch.
  \end{itemize}
\end{frame}

\begin{frame}{Next steps}
  Improve \emph{correlated uncertainties}:
  \begin{itemize}
    \item Implement confidence ellipsoids for \texttt{MvNormal} uncertainties with exact second-moments;
    \item Allow the user to tune rejection sampling;
  \end{itemize}
  \pause\medskip

  \emph{Multistage} decision rules:
  \begin{itemize}
    \item Generalize \alertt{FirstStage} to accommodate decisions $x_t$ which can only depend on \emph{observed} uncertainties $\xi_1, \ldots, \xi_t$;
    \item Will benefit from correlated uncertainties to model more complex processes.
  \end{itemize}
  \pause\medskip

  \emph{Performance}:
  \begin{itemize}
    \item Speed-up model building for larger problems.
  \end{itemize}
  \pause\bigskip

  \emph{Simulation API}:
  \begin{itemize}
    \item Provide a new \texttt{get\_decision(m, var, $\xi$)} to evaluate a primal variable on a (fully-specified) scenario.
  \end{itemize}
\end{frame}

\begin{frame}
  \begin{columns}
    \begin{column}{0.5\textwidth}
      \centering
      \LARGE GitHub Package

      \includegraphics[width=0.8\textwidth]{ldrgit.pdf}
    \end{column}
    \begin{column}{0.5\textwidth}
      \centering
      \LARGE Documentation

      \includegraphics[width=0.8\textwidth]{ldrdoc.pdf}
    \end{column}
  \end{columns}

  \centering
  \LARGE See you at the Poster Session!
\end{frame}


\end{document}


% vim:set spelllang=en:
